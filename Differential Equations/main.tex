\documentclass[journal,12pt,twocolumn]{IEEEtran}
%
\usepackage{setspace}
\usepackage{gensymb}
%\doublespacing
\singlespacing

%\usepackage{graphicx}
%\usepackage{amssymb}
%\usepackage{relsize}
\usepackage[cmex10]{amsmath}
%\usepackage{amsthm}
%\interdisplaylinepenalty=2500
%\savesymbol{iint}
%\usepackage{txfonts}
%\restoresymbol{TXF}{iint}
%\usepackage{wasysym}
\usepackage{amsthm}
%\usepackage{iithtlc}
\usepackage{mathrsfs}
\usepackage{txfonts}
\usepackage{stfloats}
\usepackage{bm}
\usepackage{cite}
\usepackage{cases}
\usepackage{subfig}
%\usepackage{xtab}
\usepackage{longtable}
\usepackage{multirow}
%\usepackage{algorithm}
%\usepackage{algpseudocode}
\usepackage{enumitem}
\usepackage{mathtools}
\usepackage{tikz}
\usepackage{circuitikz}
\usepackage{verbatim}
%\usepackage{tfrupee}
\usepackage[breaklinks=true]{hyperref}
%\usepackage{stmaryrd}
\usepackage{tkz-euclide} % loads  TikZ and tkz-base
\usetkzobj{all}
\usepackage{listings}
    \usepackage{color}                                            %%
    \usepackage{array}                                            %%
    \usepackage{longtable}                                        %%
    \usepackage{calc}                                             %%
    \usepackage{multirow}                                         %%
    \usepackage{hhline}                                           %%
    \usepackage{ifthen}                                           %%
  %optionally (for landscape tables embedded in another document): %%
    \usepackage{lscape}     
\usepackage{multicol}
\usepackage{chngcntr}
%\usepackage{enumerate}

%\usepackage{wasysym}
%\newcounter{MYtempeqncnt}
\DeclareMathOperator*{\Res}{Res}
%\renewcommand{\baselinestretch}{2}
\renewcommand\thesection{\arabic{section}}
\renewcommand\thesubsection{\thesection.\arabic{subsection}}
\renewcommand\thesubsubsection{\thesubsection.\arabic{subsubsection}}

\renewcommand\thesectiondis{\arabic{section}}
\renewcommand\thesubsectiondis{\thesectiondis.\arabic{subsection}}
\renewcommand\thesubsubsectiondis{\thesubsectiondis.\arabic{subsubsection}}

% correct bad hyphenation here
\hyphenation{op-tical net-works semi-conduc-tor}
\def\inputGnumericTable{}                                 %%

\lstset{
%language=C,
frame=single, 
breaklines=true,
columns=fullflexible
}
%\lstset{
%language=tex,
%frame=single, 
%breaklines=true
%}

\begin{document}
%


\newtheorem{theorem}{Theorem}[section]
\newtheorem{problem}{Problem}
\newtheorem{proposition}{Proposition}[section]
\newtheorem{lemma}{Lemma}[section]
\newtheorem{corollary}[theorem]{Corollary}
\newtheorem{example}{Example}[section]
\newtheorem{definition}[problem]{Definition}
%\newtheorem{thm}{Theorem}[section] 
%\newtheorem{defn}[thm]{Definition}
%\newtheorem{algorithm}{Algorithm}[section]
%\newtheorem{cor}{Corollary}
\newcommand{\BEQA}{\begin{eqnarray}}
\newcommand{\EEQA}{\end{eqnarray}}
\newcommand{\define}{\stackrel{\triangle}{=}}

\bibliographystyle{IEEEtran}
%\bibliographystyle{ieeetr}


\providecommand{\mbf}{\mathbf}
\providecommand{\pr}[1]{\ensuremath{\Pr\left(#1\right)}}
\providecommand{\qfunc}[1]{\ensuremath{Q\left(#1\right)}}
\providecommand{\sbrak}[1]{\ensuremath{{}\left[#1\right]}}
\providecommand{\lsbrak}[1]{\ensuremath{{}\left[#1\right.}}
\providecommand{\rsbrak}[1]{\ensuremath{{}\left.#1\right]}}
\providecommand{\brak}[1]{\ensuremath{\left(#1\right)}}
\providecommand{\lbrak}[1]{\ensuremath{\left(#1\right.}}
\providecommand{\rbrak}[1]{\ensuremath{\left.#1\right)}}
\providecommand{\cbrak}[1]{\ensuremath{\left\{#1\right\}}}
\providecommand{\lcbrak}[1]{\ensuremath{\left\{#1\right.}}
\providecommand{\rcbrak}[1]{\ensuremath{\left.#1\right\}}}
\theoremstyle{remark}
\newtheorem{rem}{Remark}
\newcommand{\sgn}{\mathop{\mathrm{sgn}}}
\providecommand{\abs}[1]{\left\vert#1\right\vert}
\providecommand{\res}[1]{\Res\displaylimits_{#1}} 
\providecommand{\norm}[1]{\left\lVert#1\right\rVert}
%\providecommand{\norm}[1]{\lVert#1\rVert}
\providecommand{\mtx}[1]{\mathbf{#1}}
\providecommand{\mean}[1]{E\left[ #1 \right]}
\providecommand{\fourier}{\overset{\mathcal{F}}{ \rightleftharpoons}}
%\providecommand{\hilbert}{\overset{\mathcal{H}}{ \rightleftharpoons}}
\providecommand{\system}{\overset{\mathcal{H}}{ \longleftrightarrow}}
	%\newcommand{\solution}[2]{\textbf{Solution:}{#1}}
\newcommand{\solution}{\noindent \textbf{Solution: }}
\newcommand{\cosec}{\,\text{cosec}\,}
\providecommand{\dec}[2]{\ensuremath{\overset{#1}{\underset{#2}{\gtrless}}}}
\newcommand{\myvec}[1]{\ensuremath{\begin{pmatrix}#1\end{pmatrix}}}
\newcommand{\mydet}[1]{\ensuremath{\begin{vmatrix}#1\end{vmatrix}}}
%\numberwithin{equation}{section}
\numberwithin{equation}{subsection}
%\numberwithin{problem}{section}
%\numberwithin{definition}{section}
\makeatletter
\@addtoreset{figure}{problem}
\makeatother

\let\StandardTheFigure\thefigure
\let\vec\mathbf
%\renewcommand{\thefigure}{\theproblem.\arabic{figure}}
\renewcommand{\thefigure}{\theproblem}
%\setlist[enumerate,1]{before=\renewcommand\theequation{\theenumi.\arabic{equation}}
%\counterwithin{equation}{enumi}


%\renewcommand{\theequation}{\arabic{subsection}.\arabic{equation}}

\def\putbox#1#2#3{\makebox[0in][l]{\makebox[#1][l]{}\raisebox{\baselineskip}[0in][0in]{\raisebox{#2}[0in][0in]{#3}}}}
     \def\rightbox#1{\makebox[0in][r]{#1}}
     \def\centbox#1{\makebox[0in]{#1}}
     \def\topbox#1{\raisebox{-\baselineskip}[0in][0in]{#1}}
     \def\midbox#1{\raisebox{-0.5\baselineskip}[0in][0in]{#1}}

\vspace{3cm}

\title{
%	\logo{
Differential Equations: JEE Maths
%	}
}
\author{ G V V Sharma$^{*}$% <-this % stops a space
	\thanks{*The author is with the Department
		of Electrical Engineering, Indian Institute of Technology, Hyderabad
		502285 India e-mail:  gadepall@iith.ac.in. All content in this manual is released under GNU GPL.  Free and open source.}
	
}	
%\title{
%	\logo{Matrix Analysis through Octave}{\begin{center}\includegraphics[scale=.24]{tlc}\end{center}}{}{HAMDSP}
%}


% paper titles
% can use linebreaks \\ within to get better formatting as desired
%\title{Matrix Analysis through Octave}
%
%
% author names and IEEE memberships
% note positions of commas and nonbreaking spaces ( ~ ) LaTeX will not break
% a structure at a ~ so this keeps an author's name from being broken across
% two lines.
% use \thanks{} to gain access to the first footnote area
% a separate \thanks must be used for each paragraph as LaTeX2e's \thanks
% was not built to handle multiple paragraphs
%

%\author{<-this % stops a space
%\thanks{}}
%}
% note the % following the last \IEEEmembership and also \thanks - 
% these prevent an unwanted space from occurring between the last author name
% and the end of the author line. i.e., if you had this:
% 
% \author{....lastname \thanks{...} \thanks{...} }s
%                     ^------------^------------^----Do not want these spaces!
%
% a space would be appended to the last name and could cause every name on that
% line to be shifted left slightly. This is one of those "LaTeX things". For
% instance, "\textbf{A} \textbf{B}" will typeset as "A B" not "AB". To get
% "AB" then you have to do: "\textbf{A}\textbf{B}"
% \thanks is no different in this regard, so shield the last } of each \thanks
% that ends a line with a % and do not let a space in before the next \thanks.
% Spaces after \IEEEmembership other than the last one are OK (and needed) as
% you are supposed to have spaces between the names. For what it is worth,
% this is a minor point as most people would not even notice if the said evil
% space somehow managed to creep in.



% The paper headers
%\markboth{Journal of \LaTeX\ Class Files,~Vol.~6, No.~1, January~2007}%
%{Shell \MakeLowercase{\textit{et al.}}: Bare Demo of IEEEtran.cls for Journals}
% The only time the second header will appear i/year/1963s for the odd numbered pages
% after the title page when using the twoside option.
% s
% *** Note that you probably will NOT want to include the author's ***
% *** name in the headers of peer review papers.                   ***
% You can use \ifCLASSOPTIONpeerreview for conditional compilation here if
% you desire.




% If you want to put a publisher's ID mark on the page you can do it like
% this:
%\IEEEpubid{0000--0000/00\$00.00~\copyright~2007 IEEE}
% Remember, if you use this you must call \IEEEpubidadjcol in the second
% column for its text to clear the IEEEpubid ma/year/1963rk.



% make the title area
\maketitle



%\tableofcontents

\bigskip

\renewcommand{\thefigure}{\theenumi}
\renewcommand{\thetable}{\theenumi}
%\renewcommand{\theequation}{\theenumi}

%\begin{abstract}
%%\boldmath
%In this letter, an algorithm for evaluating the exact analytical bit error rate  (BER)  for the piecewise linear (PL) combiner for  multiple relays is presented. Previous results were available only for upto three relays. The algorithm is unique in the sense that  the actual mathematical expressions, that are prohibitively large, need not be explicitly obtained. The diversity gain due to multiple relays is shown through plots of the analytical BER, well supported by simulations. 
%
%\end{abstract}
% IEEEtran.cls defaults to using nonbold math in the Abstract.
% This preserves the distinction between vectors and scalars. However,
% if the journal you are submitting to favors bold math in the abstract,
% then you can use LaTeX's standard command \boldmath ast the very start
% of the abstract to achieve this. Many IEEE journals frown on math
% in the abstract anyway.

% Note that keywords are not normally used for peerreview papers.
%\begin{IEEEkeywords}
%Cooperative diversity, decode and forward, piecewise linear
%\end{IEEEkeywords}



% For peer review papers, you can put extra information on the cover
% page as needed:
% \ifCLASSOPTIONpeerreview
% \begin{center} \bfseries EDICS Category: 3-BBND \end{center}
% \fi
%
% For peerreview papers, this IEEEtran command inserts a page break and
% creates the second title. It will be ignored for othesr modes.
%\IEEEpeerreviewmaketitle


%Download python codes using 
%\begin{lstlisting}
%svn co https://github.com/gadepall/school/trunk/ncert/computation/codes
%\end{lstlisting}

\renewcommand{\theequation}{\theenumi}
\begin{enumerate}[label=\arabic*.,ref=\theenumi]
%\begin{enumerate}[label=\arabic*.,ref=\thesubsection.\theenumi]
\numberwithin{equation}{enumi}
\item A solution of the following diffrential equation is
\begin{align*}
\left(\frac{dy}{dx}\right)^{2} - x\frac{dy}{dx} + y = 0
\end{align*}
\begin{enumerate}
\item y = 2
\item y = 2x
\item y = 2x - 4
\item $y = 2x^2 - 4$
\end{enumerate}

\item If $x^2 + y^2 = 1$, then
\begin{enumerate}
\item $yy'' - 2(y')^{2} + 1 = 0$
\item $yy'' + (y')^{2} + 1 = 0$
\item $yy'' + (y')^{2} - 1 = 0$
\item $yy'' + 2(y')^{2} + 1 = 0$
\end{enumerate}

\item If $y(t)$ is a solution of
\begin{align*}
(1 + t)\frac{dy}{dt} - ty = 1, y(0) = -1
\end{align*}
then $y(1)$ is equal to
\begin{enumerate}
\item -1/2
\item e + 1/2
\item e - 1/2
\item 1/2
\end{enumerate}

\item If $y = f(x)$ and $\frac{2 + \sin x}{y + 1}\left(\frac{dy}{dx}\right) = \cos x$
y(0) = -1, then $y\left(\frac{\pi}{2}\right)$ equals
\begin{enumerate}
\item 1/3
\item 2/3
\item -1/3
\item 1
\end{enumerate}

\item If $y = f(x)$ and it is followsthe relation $x\cos y + y\cos x = \pi$ then $y''(0)$ = 
\begin{enumerate}
\item 1
\item -1
\item $\pi - 1$
\item $\pi$
\end{enumerate}

\item The solution of primitive integral equation $(x^2 + y^2)dy = xydx$ is $y = y(x)$. If $y(1) = 1$ and $(x_0 = e)$, then $x_0$ is equal to
\begin{enumerate}
\item $\sqrt{2(e^2 - 1)}$
\item $\sqrt{2(e^2 + 1)}$
\item $\sqrt{3}e$
\item $\sqrt{\frac{e^2 + 1}{2}}$
\end{enumerate}

\item For the primitive integral equation $ydx + y^2dy = x$; $x \in R$, $y > 0$, $y = y(x)$, $y(1) = 1$, then $y(-3)$ is
\begin{enumerate}
\item 3
\item 2
\item 1
\item 5
\end{enumerate}

\item The differential equation 
\begin{align*}
\frac{dy}{dx} = \frac{\sqrt{1 - y^2}}{y}
\end{align*}
determines a family of circles with
\begin{enumerate}
\item variable radii and a fixed centre at (0, 1)
\item variable radii and a fixed centre at (0, -1)
\item fixed radius 1 and variables centres along the x-axis
\item fixed radius 1 and variables centres along the y-axis
\end{enumerate}

\item The function $y = f(x)$ is the solution of the differential equation
\begin{align*}
\frac{dy}{dx} = \frac{xy}{x^2 - 1} = \frac{x^4 + 2x}{\sqrt{1 - x^2}}
\end{align*}
in (1, -1) satisfying f(0) = 0. Then 
\begin{align*}
\int_{-\sqrt{3}/2}^{\sqrt{3}/2}f(x)dx
\end{align*}
is
\begin{enumerate}
\item $\frac{\pi}{3} - \frac{\sqrt{3}}{2}$
\item $\frac{\pi}{3} - \frac{\sqrt{3}}{4}$
\item $\frac{\pi}{6} - \frac{\sqrt{3}}{4}$
\item $\frac{\pi}{6} - \frac{\sqrt{3}}{2}$ 
\end{enumerate}

\item If $y = y(x)$ satisfies the differential equation
\begin{align*}
8\sqrt{x}\left(\sqrt{9 + \sqrt{x}}\right)dy = \left(\sqrt{4 + \sqrt{9 + \sqrt{x}}}\right)^{-1}dx, x > 0
\end{align*}
and $y(0) = \sqrt{7}$, then $y(256)$ = 
\begin{enumerate}
\item 3
\item 9
\item 16
\item 80
\end{enumerate}

\textbf{MCQs with One or More than One Correct}

\item The order of the differential equation whose general solution is given by
\begin{align*}
y = (C_1 + C_2)\cos(x + C_3) - C_4e^x + C_5
\end{align*}
where $C_1, C_2, C_3, C_4, C_5$ are arbitrary constants, is
\begin{enumerate}
\item 5
\item 4
\item 3
\item 2
\end{enumerate}

\item The differential equation representing the family of curves
\begin{align*}
y = 2c\left(x + \sqrt{x}\right)
\end{align*}
where c is a positive parameter, is of
\begin{enumerate}
\item order 1
\item order 2
\item degree 3
\item degree 4
\end{enumerate}

\item A curve $y = f(x)$ passes through (1, 1) and at P(x, y), tangent cuts the x-axis and y-axis at A and B respectively such that BP : AP = 3 : 1. then
\begin{enumerate}
\item equation of curve is $xy' - 3y = 0$
\item normal at (1, 1) is $x + 3y = 4$
\item curve passes through (2, 1/8)
\item equation of curve is $xy' + 3y = 0$
\end{enumerate}

\item If $y(x)$ satisfies the differential equation 
\begin{align*}
y' = y\tan x = 2x\sec x 
\end{align*}
and $y(0) = 0$, then
\begin{enumerate}
\item $y\left(\frac{\pi}{4}\right) = \frac{\pi^2}{8\sqrt{2}}$
\item $y'\left(\frac{\pi}{4}\right) = \frac{\pi^2}{18}$
\item $y\left(\frac{\pi}{3}\right) = \frac{\pi^2}{9}$
\item $y'\left(\frac{\pi}{3}\right) =  y'\left(\frac{4\pi}{3}\right)+ \frac{2\pi^2}{3\sqrt{3}}$
\end{enumerate}

\item A curve passes through the point $\left(1, \frac{\pi}{6}\right)$. Let the slope of the curve at each point (x, y) be
\begin{align*}
\frac{y}{x} + \sec\left(\frac{y}{x}\right), x > 0
\end{align*}
Then the equation of the curve is
\begin{enumerate}
\item $\sin\left(\frac{y}{x}\right) = log x + \frac{1}{2}$
\item $\cosec\left(\frac{y}{x}\right) = log x + \frac{1}{2}$
\item $\sec\left(\frac{2y}{x}\right) = log x + \frac{1}{2}$
\item $\cos\left(\frac{2y}{x}\right) = log x + \frac{1}{2}$
\end{enumerate}

\item Let $f(x)$ be a solution of the differential equation
\begin{align*}
(1 + e^x)y' + ye^x = 1
\end{align*}
If $y(0) = 2$, then which of the following statementis (are) true?
\begin{enumerate}
\item $y(-4) = 0$
\item $y(-2) = 0$
\item $y(x)$ has a critical point in the interval (-1, 0)
\item $y(x)$ has no critical point in the interval (-1, 0)
\end{enumerate}

\item Consider the family of all circles whose centres lie on the straight line y = x. If this family of circle is represented by the differential equation $Py'' + Qy' + 1 = 0$, where P, Q are functions of x, y and $y'$, then which of the following statement is(are) true?
\begin{enumerate}
\item P = y + x
\item P = y - x
\item $P + Q = 1 - x + y + y' + (y')^{2}$
\item $P - Q = x + y - y' - (y')^{2}$
\end{enumerate}

\item Let $f: (0, \infty) \to R$ be a differentiable function such that
\begin{align*}
f'(x) = 2 - \frac{f(x)}{x}
\end{align*}
for all $x \in (0, \infty)$ and $f(1) \neq 1$. Then 
\begin{enumerate}
\item $\lim_{x \to 0^{+}}f'\left(\frac{1}{x}\right) = 1$
\item $\lim_{x \to 0^{+}}xf'\left(\frac{1}{x}\right) = 2$
\item $\lim_{x \to 0^{+}}x^2f'(x) = 0$
\item $|f(x)| \leq 2$ for all $x \in (0, 2)$
\end{enumerate}

\item A solution curve of the following differential equation
\begin{align*}
(x^2 + xy + 4x + 2y + 4)\frac{dy}{dx} - y^2 = 0, x > 0
\end{align*}
passes through the point (1, 3). Then the solution curve
\begin{enumerate}
\item intersects y = x + 2 exactly at one point
\item intersects y = x + 2 exactly at two points
\item intersects $y = (x + 2)^2$
\item does NOT intersect $y = (x + 2)^2$
\end{enumerate}

\item Let $f: [0, \infty) \to R$ be a continuous function such that
\begin{align*}
f(x) = 1 - 2x + \int_{0}^{x}e^{x - t}f(t)dt
\end{align*}
for all $x \in [0, \infty)$. Then which of the following statement(s) is(are) true?
\begin{enumerate}
\item The curve $y = f(x)$ passes through the point (1, 2)
\item The curve $y = f(x)$ passes through the point (2, -1)
\item The area of the region
\begin{align*}
\left\lbrace(x, y) \in [0, 1] \times R: f(x) \leq y \leq \sqrt{1 - x^2}\right\rbrace
\end{align*}
is $\frac{\pi - 2}{4}$
\item The area of the region
\begin{align*}
\left\lbrace(x, y) \in [0, 1] \times R: f(x) \leq y \leq \sqrt{1 - x^2}\right\rbrace
\end{align*}
is $\frac{\pi - 1}{4}$
\end{enumerate}

\item Let $\Gamma$ denotes a curve $y = f(x)$ which is in the first quadrant and let the point (1, 0) lie on it. Let the tangent to $\Gamma$ at a point P intersects the y-axis at $Y_p$. If $PY_p$ has length 1 for each point P on $\Gamma$, then which of the following option(s) is(are) correct?
\begin{enumerate}
\item $y = -log_e\left(\frac{1 + \sqrt{1 - x^2}}{x}\right) + \sqrt{1 - x^2}$
\item $xy' - \sqrt{1 - x^2} = 0$
\item $y = log_e\left(\frac{1 + \sqrt{1 - x^2}}{x}\right) - \sqrt{1 - x^2}$
\item $xy' + \sqrt{1 - x^2} = 0$
\end{enumerate} 

\textbf{Subjective Problems}

\item If $(a + bx)e^{y/x} = x$, then prove that
\begin{align*}
x^3\frac{d^{2}y}{dx^{2}} = \left(x\frac{dy}{dx} - y\right)
\end{align*}

\item A normal is drawn at apoint P(x, y) of a curve. It meets the x-axis at Q. If PQ is of constant length k, then show that the differential equation describing such curve is
\begin{align*}
y \frac{dy}{dx} = \pm \sqrt{k^2 - y^2}
\end{align*}
Find the equation of such curve passing through (0, k).

\item Let $y = f(x)$ be a curve passing through (1, 1) such that the triangle formed by the coordinate axes and the tangent at any point of the curve lies in the first quadrant and has area 2. Form the differential equation and determine all such possible curves.

\item Determine the equation of the curve passing through the origin, in the form $y = f(x)$, which satisfies the differential equation
\begin{align*}
\frac{dy}{dx} = \sin(10x+ 6y)
\end{align*}

\item Let $u(x)$ and $v(x)$ satisfy the differential equation
\begin{align*}
\frac{du}{dx} + p(x)u = f(x)
\end{align*}
\begin{align*}
\frac{dv}{dx} + p(x)v = g(x)
\end{align*}
where $p(x)$, $f(x)$ and $g(x)$ are continuous functions. If $u(x_1) > v(x_1)$ for some $x_1$ and $f(x) > g(x)$ for all $x > x_1$, prove that any point (x, y) where $x > x_1$ does not satisfy the equation $y = u(x)$ and $y = v(x)$ 


\item A curve passing through the point (1, 1) has the property that the perpendicular distance of the origin from the normal at any point P of the curve is equal to the distance of P from the x-axis. Determine the equation of the curve.

\item A country has a food deficit of 10 percentage. Its population grows continuously at a rate of 3 percentage per year. Its annual food production every year is 4 percentage more than that of the last year. Assuming that the average food requirement per person remains constant, prove that the country will become self-sufficient in food after n years, where n is the smallest integer bigger than or equal to $\frac{ln 10 - ln 9}{ln(1.04) - 0.03}$.

\item A hemispherical tank of radius 2 metres is initially full of water and has an outlet of $12cm^{2}$ cross-sectional area at the bottom. The outlet is opened at some instant. The flow through the outlet is according to the law 
\begin{align*}
v(t) = 0.6\sqrt{2gh(t)}
\end{align*}
where $v(t)$ and $h(t)$ are respectively the velocity of the flow through the outlet and height of the water level above the outlet at time t, and g is the accelration due to gravity. Find the time it takes to empty the tank.

\item A right circular cone with radius R and height H contains a liquid which evaporates at a rate proportional to its surface area in contact with air(proportional constant = $k > 0$). Find the time after which the cone is empty.

\item A curve $'C'$ passes through (2, 0) and the slope at (x, y) as 
\begin{align*}
\frac{(x + 1)^2 + (y - 3)}{x + 1}
\end{align*}
Find the equation of the curve. Find the area bounded by the curve and x-axis in fourth quadrant.

\item If length of tangent at any point on the curve $y = f(x)$ intercepted berween the point and the x-axis os of length 1. Find the equation of the curve.

\textbf{Section-B}

\item The order and degree of the differential equation
\begin{align*}
\left(1 + 3\frac{dy}{dx}\right)^{2/3} = 4\frac{d^{3}y}{dx^{3}}
\end{align*}
 are
\begin{enumerate}
\item $\left(1, \frac{2}{3}\right)$
\item (3, 1)
\item (3, 3)
\item (1, 2) 
\end{enumerate}

\item The solution of the equation
\begin{align*}
\frac{d^{2}y}{dx^{2}} = 2e^{-x}
\end{align*}
\begin{enumerate}
\item $\frac{e^{-2x}}{4}$
\item $\frac{e^{-2x}}{4} + cx + d$
\item $\frac{1}{4}e^{-2x} + cx^{2} + d$
\item $\frac{1}{4}e^{-4x} + cx + d$
\end{enumerate}

\item The degree and order of the differential equation of the family of all parabolas whose axis is x-axis, are respectively.
\begin{enumerate}
\item 2, 3
\item 2, 1
\item 1, 2
\item 3, 2
\end{enumerate}

\item The solution of the differential equation is
\begin{align*}
(1 + y^{2}) + (x - e^{\tan^{-1}y})\frac{dy}{dx} = 0
\end{align*}
\begin{enumerate}
\item $xe^{2\tan^{-1}y} = e^{\tan^{-1}y} + k$
\item $(x - 2) = ke^{2\tan^{-1}y}$
\item $2xe^{2\tan^{-1}y} = e^{2\tan^{-1}y} + k$
\item $xe^{\tan^{-1}y} = e^{\tan^{-1}y} + k$
\end{enumerate}

\item The differential equation for the family of circle
\begin{align}
x^2 + y^2 -2ay = 0
\end{align}
where a is an arbitraty constant is
\begin{enumerate}
\item $(x^2 + y^2)y' = 2xy$
\item $2(x^2 + y^2)y' = xy$
\item $(x^2 - y^2)y' = 2xy$
\item $2(x^2 - y^2)y' = xy$
\end{enumerate}

\item Solution of the differential equation is
\begin{align*}
ydx + (x + x^2y)dy = 0
\end{align*}
\begin{enumerate}
\item $logy = Cx$
\item $-\frac{1}{xy} + log y = C$
\item $\frac{1}{xy} + logy = C$
\item $-\frac{1}{xy} = C$
\end{enumerate}

\item The differential equation representing the family of curves
\begin{align*}
y^2 = 2c(x + \sqrt{c}), c > 0 
\end{align*}
where c is a parameter, is of order and degree follows:
\begin{enumerate}
\item order 1, degree 2
\item order 1, degree 1
\item order 1, degree 3
\item order 2, degree 2
\end{enumerate}

\item If 
\begin{align*}
x\frac{dy}{dx} = y(logy - logx + 1)
\end{align*}
then the solution of the differential equation is
\begin{enumerate}
\item $ylog\left(\frac{x}{y}\right) = cx$
\item $xlog\left(\frac{y}{x}\right) = cy$
\item $log\left(\frac{y}{x}\right) = cx$
\item $log\left(\frac{x}{y}\right) = cy$
\end{enumerate} 

\item The diffrential equation whose solution is
\begin{align}
Ax^2 + By^2 = 1
\end{align}
where A and B are arbitrary constants is of
\begin{enumerate}
\item second order and second degree
\item first order and second degree
\item first order and first degree
\item second order and first degree
\end{enumerate}

\item The differential equation of all circles passing through the origin and having their centres on the x-axis is
\begin{enumerate}
\item $y^2 = x^2 + 2xy\frac{dy}{dx}$
\item $y^2 = x^2 - 2xy\frac{dy}{dx}$
\item $x^2 = y^2 + xy\frac{dy}{dx}$
\item $x^2 = y^2 + 3xy\frac{dy}{dx}$
\end{enumerate}

\item The solution of the diffrential equation
\begin{align*}
\frac{dy}{dx} = \frac{x + y}{x}
\end{align*}
satisfying the condition $y(1) = 1$ is
\begin{enumerate}
\item $y = lnx + x$
\item $y = xlnx + x^2$
\item $y = xe^{x - 1}$
\item $y = xlnx + x$
\end{enumerate}

\item The differential equation which represents the family of curves $y = c_1e^{c_2x}$ where $c_1$ and $c_2$ are arbitrary constants, is
\begin{enumerate}
\item $y'' = y'y$
\item $yy'' = y'$
\item $yy'' = (y')^2$
\item $y' = y^2$
\end{enumerate} 

\item Solution of the differential equation is
\begin{align*}
\cos xdy = y(\sin x - y)dx, 0 < x < \frac{\pi}{2}
\end{align*}
\begin{enumerate}
\item $y\sec x = \tan x + c$
\item $y\tan x = \sec x + c$
\item $\tan x = (\sec x + c)y$
\item $\sec x = (\tan x + c)y$
\end{enumerate}

\item If 
\begin{align*}
\frac{dy}{dx} = y + 3 > 0, y(0) = 2
\end{align*}
then $yln(2)$ is equal to
\begin{enumerate}
\item 5
\item 13
\item -2
\item 7
\end{enumerate}

\item Let I be the purchase value of an equipment ans $V(t)$ be the value after it has been used for t years. The value 
$V(t)$ depreciates at a rate of given by differential equation 
\begin{align*}
\frac{dV(t)}{dt} = -k(T - t), k > 0
\end{align*}
where k is constant and T is the total life in years of the eqipment. Then the scrap value $V(T)$ of the equipment is
\begin{enumerate}
\item $I - \frac{kT^2}{2}$
\item $I - \frac{k(T - t)^2}{2}$
\item $e^{-kT}$
\item $T^2 - \frac{1}{k}$
\end{enumerate} 

\item The population $p(t)$ at time t of a certain mouse species satisfies the differential equation
\begin{align*}
\frac{dp(t)}{dt} = 0.5p(t) - 450
\end{align*}
If $p(0) = 850$, then the time at which the population becomes zero is
\begin{enumerate}
\item $2ln18$
\item $ln9$
\item $\frac{1}{2}ln18$
\item $ln18$
\end{enumerate}

\item At present, afirm is manufacturing 2000 items. It is estimated that the rate of chnage of production P w.r.t. additional number of workers x is given by
\begin{align*}
\frac{dP}{dx} = 100  - 12\sqrt{x}
\end{align*}
If the firm employs 25 more workers, then the new value of production of items is
\begin{enumerate}
\item 2500
\item 3000
\item 3500
\item 4500
\end{enumerate}

\item Let the population of rabbits surviving at time t be governed by the differential equation
\begin{align*}
\frac{dp(t)}{dt} = \frac{1}{2}p(t) - 200.
\end{align*}
If $p(0) = 100$, then $p(t)$ equals
\begin{enumerate}
\item $600 - 500e^{t/2}$
\item $400 - 300e^{-t/2}$
\item $400 - 300e^{t/2}$
\item $300 - 200e^{-t/2}$
\end{enumerate}

\item Let $y(x)$ be the solution of the differential equation
\begin{align*}
(xlogx)\frac{dy}{dx} + y = 2x logx, (x \geq 1)
\end{align*}
Then $y(e)$ is equal to
\begin{enumerate}
\item 2
\item 2e
\item e
\item 0
\end{enumerate}

\item If a curve $y = f(x)$ passes through the point (1, -1) and satisfies the differential equation 
\begin{align*}
y(1 + xy)dx = xdy
\end{align*}
then, $f\left(-\frac{1}{2}\right)$ is equal to
\begin{enumerate}
\item $\frac{2}{5}$
\item $\frac{4}{5}$
\item $-\frac{2}{5}$
\item $-\frac{4}{5}$
\end{enumerate}

\item If
\begin{align*}
(2 + \sin x)\frac{dy}{dx} + (y + 1)\cos x = 0, y(0) = 1
\end{align*}
then $y\left(\frac{\pi}{2}\right)$ is equal to
\begin{enumerate}
\item 4/3
\item 1/3
\item -2/3
\item -1/3
\end{enumerate}

\item Let $y - y(x)$ be the solution of the differential equation
\begin{align*}
\sin x\frac{dy}{dx} + y\cos x = 4x, x \in (0, \pi)
\end{align*}
If $y\left(\frac{\pi}{2}\right) = 0$, then $y\left(\frac{\pi}{6}\right)$ is equal to
\begin{enumerate}
\item $\frac{-8}{9\sqrt{3}}\pi^{2}$
\item $\frac{-8}{9}\pi^{2}$
\item $\frac{-4}{9}\pi^{2}$
\item $\frac{4}{9\sqrt{3}}\pi^{2}$
\end{enumerate}

\item If $y = y(x)$ is the solution of the differential equation 
\begin{align*}
x\frac{dy}{dx} + 2y = x^2
\end{align*}
satisfying $y(a) = 1$, then $y\left(\frac{1}{2}\right)$ is equal to
\begin{enumerate}
\item $\frac{7}{64}$
\item $\frac{1}{4}$
\item $\frac{49}{16}$
\item $\frac{13}{16}$
\end{enumerate}

\item The solution of the differential equation 
\begin{align*}
x\frac{dy}{dx} + 2y = x^2 (x \neq 0)
\end{align*}
with $y(1) = 1$ is
\begin{enumerate}
\item $y = \frac{4}{5}x^3 + \frac{1}{5x^2}$
\item $y = \frac{1}{5}x^3 + \frac{1}{5x^2}$
\item $y = \frac{1}{4}x^2 + \frac{3}{4x^2}$
\item $y = \frac{3}{4}x^2 + \frac{1}{4x^2}$
\end{enumerate}

\textbf{Assertion and Reason Type Questions}

\item Let a solution $y = y(x)$ of the differential equation
\begin{align*}
x\sqrt{x^2 - 1}dy - y\sqrt{y^2 - 1}dx = 0
\end{align*}
satisfy $y(2) = \frac{2}{\sqrt{3}}$.

\textbf{Statement-1:}

\begin{align*}
y(x) = \sec\left(\sec^{-1}x - \frac{\pi}{6}\right)
\end{align*}

\textbf{Statement-2:} 

\begin{align*}
y(x): \frac{1}{y} = \frac{2\sqrt{3}}{x} - \sqrt{1 - \frac{1}{x^2}}
\end{align*}
\begin{enumerate}
\item Statement-1 is true, Statement-2 is true, Statement-2 is a correct explanation for Statement-2
\item Statement-1 is true, Statement-2 is true, Statement-2 is not a correct explanation for Statement-2
\item Statement-1 is true, Statement-2 is false
\item Statement-1 is false, Statement-2 is true
\end{enumerate}

\textbf{Integer Value Correct Type}

\item Let
\begin{align*}
y'(x) + y(x)g'(x) = g(x), g'(x), y(0) = 0, x \in R 
\end{align*}
where $f'(x)$ denotes $\frac{df(x)}{dx}$ and $g(x)$ is a given non-constant differentiable function on R with $g(0) = g(2) = 0$. Then the value of $g(2)$ is

\item Let $f: R \to R$ be a differentiable function with $f(0) = 0$. If $y = f(x)$ satisfies the differential equation
\begin{align*}
\frac{dy}{dx} = 2(2 + 5y)(5y - 2)
\end{align*}  
then the value of $\lim_{x \to \infty}f(x)$ is..........

\item Let $f: R \to R$ be a differentiable function with $f(0) = 1$ and satisfies the equation 
\begin{align*}
f(x + y) = f(x)f'(y) + f'(x)f(y), y \in R 
\end{align*}
Then, the value of $log_e(f(4))$ is..............

\clearpage

\textbf{Match the Following Questions:}

\item Match the following
\begin{table}[ht!]
\centering
\begin{tabular}{c c} 
 \textbf{Column I} & \textbf{Column II}\\ [0.5ex] 
 (A) Interval contained in the\\ domain of
     non-zero solutions\\ of the differential equation\\
     $(x - 3)^2 + y' + y = 0$                                     &(p) $\left(-\frac{\pi}{2}, \frac{\pi}{2}\right)$\\ 
 (B) Interval containing the\\ value of the integral\\
     $\int_{1}^{5}(x-1)(x-2)(x-3)(x-4)$\\ $(x-5)dx$                    &(q) $\left(0, \frac{\pi}{2}\right)$\\
 (C) Interval in which at least\\ one of the points of
     local\\ maximum of $\cos^{2} + \sin x$ lies                    &(r) $\left(\frac{\pi}{8}, \frac{5\pi}{4}\right)$\\                                                                     
 (D) The Interval in which\\ $\tan^{-1}(\sin x + \cos x)$ is            &(s) $\left(0, \frac{\pi}{8}\right)$\\[1ex] 
\end{tabular}
\end{table}\\















































\end{enumerate}

\end{document}


